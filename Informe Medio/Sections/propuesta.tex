\subsection{Objetivo del Banco}

\qquad El objetivo de la empresa es lograr tener una opcion de prepago para los créditos hipotecarios. Una opción es un tipo de contrato que da el derecho, pero no la obligación, de comprar o vender un activo subyacente, que en este caso corresponde al crédito. Es decir, ofrecerle al cliente que compre esta opción al banco y asi poder prepagar cuando el estime conveniente sin que el banco le cobre la clausula de prepago mencionada anteriormente.

\qquad Se reconocen tres razones de por que es relevante que ITAÚ pueda ofrecer este tipo de contratos para sus créditos:

\begin{itemize}
    \item Se podrán ofrecer mejores tasas de interes. El mercado de los créditos hipotecarios es competitivo, por lo que es sumamente valioso poder ajustar a cuanto se ofrecen los productos. Esto será posible debido a que la opción ya esta cubriendo el riesgo del prepago, por lo que no es necesario aplicar la heuristica anteriormente descrita y no subir innecesariamente las tasas de interes.
    \item Las opciones se pueden aplicar de distintas maneras, esto aumentará considerablemente la variedad de productos que tiene el banco tiene para ofrecer, ajustandose a las necesidades de cada cliente.
    \item Gestion de riesgos.
\end{itemize}

\subsection{Propuesta y Objetivo del Proyecto}

\qquad El objetivo principal de este proyecto corresponde a \textbf{valorizar opcionalidades de prepago como una opción del deudor sobre el crédito}. Lo cual, de forma más sencilla, corresponde a cuantificar cuanto le cuesta al banco el prepago del cliente en un tiempo futuro.

\qquad Es por esto que la propuesta de nuestro proyecto es \textbf{calibrar un modelo discreto de tasas para valorizar la opcionalidad de prepago en créditos hipotecarios}. Cuyos detalles se irán describiendo a lo largo de este informe.