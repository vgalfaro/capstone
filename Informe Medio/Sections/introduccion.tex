\qquad El presente informe de avance corresponde al proyecto de valorización de las opcionalidades de prepago en créditos hipotecarios otorgado por el banco Itaú. Se detallan los avances realizados hasta la fecha, incluyendo el estudio preliminar de los datos, el procesamiento de estos, el análisis exploratorio, los modelos a desarrollar y la planificación de tareas futuras.

\subsection{Contexto Empresa}

\qquad El Banco Itaú Chile es una de las principales instituciones financieras del país y forma parte del grupo Itaú Unibanco, el mayor banco privado de América Latina, con sede en Brasil. En Chile, Itaú inició sus operaciones en 2006, consolidando su presencia tras la fusión con CorpBanca en 2016, lo que lo posicionó entre los bancos más relevantes del sistema financiero nacional.

\qquad El banco ofrece una amplia gama de servicios financieros tanto a personas como a empresas. Sin embargo, aquí nos centramos en su rol de banco de retail, en el cual se centra en ofrecer una diversa variedad de productos y servicios financieros accesibles, como cuentas corrientes, créditos de consumo, hipotecarios, tarjetas de crédito, seguros e inversiones, junto con herramientas digitales que facilitan la gestión cotidiana del dinero.

\subsection{Presentación Problema}