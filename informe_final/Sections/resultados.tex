
\subsection{Calibración del Modelo de Tasas}
Antes de valorar la opción, es fundamental validar que el modelo de tasas replica correctamente las condiciones del mercado. 
\subsubsection{Construcción del Árbol de Ho--Lee}
Se implementó el árbol binomial de Ho--Lee calibrado a la curva Swap Cámara Promedio vigente al 15 de septiembre de 2025. 

\begin{figure}[H]
    \centering
        \includegraphics[scale=0.5]{images/holee_hist.png}
    \caption{Árbol de tasas cortas generado por el modelo de Ho-Lee a 5 años plazo. La línea azul representa el valor esperado de la tasa en el plazo t.
    }\label{fig:holeetreestd}
\end{figure}

A diferencia de modelos normales simples que pueden arrojar tasas negativas con alta probabilidad en plazos largos, la calibración realizada con la volatilidad actual ($\sigma\approx$10.06 bps, calculada vía desviación estándar de la muestra solo en 2025) muestra un comportamiento robusto. Como se observa en la Figura \ref{fig:holeetreestd}, la dispersión de tasas a 5 años se mantiene en rangos coherentes con la historia reciente (entre 3.5\% y 7.0\%).

\subsubsection{Impacto de la Volatilidad en la Estructura del Árbol}

Para evaluar la sensibilidad del modelo a la estimación de la volatilidad, se generó un escenario alternativo con una volatilidad histórica de mayor plazo ($\sigma\approx$16.84 bps, calculada vía EWMA desde 2021). Es posible observar como el árbol se ``abre'' significativamente más, capturando escenarios extremos para tasas más altas y bajas (ver Figura \ref{fig:holeetreeewma}), 

\begin{figure}[H]
    \centering
        \includegraphics[scale=0.5]{images/holee_ewma.png}
    \caption{Árbol de tasas bajo un escenario de mayor volatilidad. Se aprecia una mayor dispersión de los nodos terminales.
    }\label{fig:holeetreeewma}
\end{figure}

\subsection{Valoración de la Opción de Prepago}

Utilizando el árbol de Ho-Lee calibrado, se procedió a valorar la opción de prepago, considerando las características específicas del bono en cuestión (cupón, vencimiento, frecuencia de pago, etc.). Para el estudio del prepago utilizamos un crédito con las siguientes características, pues es representativo de la cartera coporativa del banco:
\begin{itemize}
    \item Capital: 100.000.000 CLP.
    \item Tipo de crédito: Francés (Cuotas iguales).
    \item Tasa contractual: 4.6\% anual.
    \item Plazo: 5 años.
    \item Volatilidad del árbol de Ho--Lee: 16.8 bps.
\end{itemize}

\subsubsection{Dinámica del Valor de la Opción en el Tiempo}
La Figura \ref{fig:opcfrances56} muestra el lattice del valor de la opción de prepago (bps sobre el capital) en cada nodo del árbol de Ho--Lee, es decir, en cada posible escenario de la tasa corta.
\begin{figure}[H]
    \centering
        \includegraphics[scale=0.5]{images/opcfrances56.png}
    \caption{Lattice del valor de la opción de prepago para un crédito a 5 años. Cada nodo representa el valor de la opción en ese estado y tiempo específico.
    }\label{fig:opcfrances56}
\end{figure}
Podemos observar tres fenómenos relevantes:
\begin{enumerate}
    \item Convergencia a cero: Al final del crédito, el valor de la opción es cero, cumpliendo con la condición de frontera. Esto valida la correctitud del algoritmo implementado. 
    \item Valor máximo: El valor de la opción no es máximo al inicio, sino que alcanza su \textit{peak} alrededor de la vida media del crédito. Esto se debe a que al inicio la probabilidad acumulada de que las tasas bajen significativamente es menor. A medida que avanza el tiempo y la volatilidad dispersa las tasas, la probabildiad de entrar a escenarios favorables para el prepago aumenta, elevando el valor de la opción antes de decaer por la amortización del capital.
    \item Magnitud del costo: Para este escenario, el modelo estima un costo esperado de prepago en el día 0 de 68.17 bps sobre el monto del crédito. Esto monto sería el que el banco debería considerar como ajuste en la tasa ofrecida para compensar el riesgo de prepago.
\end{enumerate}

\subsection{Análisis de Sensibilidad}

Un hallazgo crítico de los resultados es la alta sensibilidad del valor de la opción ante la volatilidad del mercado. 

Al aumentar la volatilidad del modelo a 100 bps, el valor de la opción de prepago casi se duplica, pasando de 68 bps a 120.99 bps (ver Figura \ref{fig:opcfrances100}), es decir, un aumento de un 1.20\% en la tasa ofrecida al cliente para compensar el riesgo de prepago. 

\begin{figure}[H]
    \centering
        \includegraphics[scale=0.5]{images/opcfrances100.png}
    \caption{Valoración de la opción bajo un escenario de aumento de la volatilidad ($\sigma=100$ bps).
    }\label{fig:opcfrances100}
\end{figure}

Este resultado demuestra que el métdo actualde cobrar un \textit{spread} fijo o una comoisión basada en meses de interés (1.5 meses) es ineficiente:
\begin{itemize}
    \item En periodos de baja volatilidad, el banco podría estar cobrando de más, reduciendo su competitividad en el mercado.
    \item En periodos de alta volatilidad, el banco podría estar subvalorando el riesgo, asumiendo pérdidas potenciales significativas en caso de prepagos masivos.
\end{itemize}


\subsection{Entregables}
Como cierre del proceso de desarrollo y validación, se consolidan los siguientes entregables acordados con la contraparte del Banco Itaú, los cuales materializan la solución propuesta:
\begin{enumerate}
    \item Paquete de datos para la calibración de los modelos: Conjunto de datos que contiene las curvas Swap Cámara Promedio procesadas y transformadas a curvas cupón cero.
    \item Librería de modelos de tasa en Python: Implementación del modelo de Ho--Lee, estructurado de manera modular. El código permite la construcción del árbol binomial calibrado a una curva de mercado arbitraria y a una volatilidad definida por el usuario.
    \item Motor de valoración de opciones en Python: Script que implementa el algoritmo de valoración de opciones de prepago utilizando el árbol de Ho--Lee. Incluye funciones para definir las características del crédito (francés, alemán y bullet) y calcular el valor de la opción en cada nodo del árbol, retornando el valor inicial de la opción en bps sobre el capital.
    \item README detallado: Documento que explica el uso de las librerías desarrolladas. Incluye ejemplos de como cargar datos, calibrar el modelo y valorar opciones de prepago para distintos tipos de créditos. También, ejemplos para la generación de distintos gráficos de análisis.
\end{enumerate}

Estos entregables pueden ser encontrados en el siguiente repositorio de GitHub: \href{https://github.com/vgalfaro/capstone.git}{Valoración de Opcionalidades de Prepago en Créditos Hipotecarios}.
