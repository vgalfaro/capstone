\subsection{Conclusiones}
A partir del desarrollo e implementación del modelo de valoración de opciones de prepago, se derivan las siguientes conclusiones principales: 
\begin{enumerate}
    \item Viabilidad técnica y práctica: Se validó la factibilidad de implementar un modelo de tasa corta basado en árboles binomiales (Ho--Lee) para el mercado chileno. El modelo fue calibrado exitosamente a la curva Swap Cámara Promedio, permitiendo valorar el prepago de manera consistente. 
    \item Subestimación del riesgo en la práctica actual: Los resultados cuantitativos revelan que la comisión actual subestima el valor real de la opción de prepago en los escenarios analizados. Esto sugiere que Itaú podría estar asumiendo un riesgo mayor al anticipado.
    \item Alta sensibilidad a la volatilidad: Se demostró que el precio de la opción es altamente sensible al parámetro de volatilidad. Al estresar el modelo, el costo del prepago puede incrementarse significativamente, lo que confirma la necesidad de una gestión dinámica de este riesgo, pues cobrar un \textit{spread} fijo es ineficiente tanto en periodos de baja como de alta volatilidad.
\end{enumerate}
\subsection{Limitaciones del Modelo Actual}

A pesar de los resultados observados, el modelo presenta restricciones 
inherentes a su formulación. En primer lugar, se asume una volatilidad 
constante, lo cual no refleja la naturaleza dinámica del mercado. Además, 
el modelo de un solo factor limita la capacidad de capturar riesgos 
específicos del banco, como el costo de fondeo. Estas simplificaciones 
reducen la precisión del modelo y su aplicabilidad en escenarios más 
complejos.

\subsection{Trabajo Futuro}

Para superar las limitaciones identificadas y encaminar la solución hacia una herramienta más productiva, se proponen las siguientes líneas de trabajo futuro:
\begin{itemize}
    \item Implementar el modelo Ho--Lee extendido para mejorar la representación de la estructura temporal de tasas, permitiendo modelar la volatilidad como una variable en el tiempo.
    \item Incorporar más tipos de créditos (alemán, bullet) para ampliar la cobertura del análisis y adaptarse a la diversidad de productos ofrecidos por el banco.
    \item Migrar a un modelo de dos factores para capturar tanto el riesgo de mercado como el riesgo Itaú, específicamente el costo de fondeo, enriqueciendo la evaluación del prepago.
    \item Analizar la estabilidad del precio de la opción utilizando volatilidad histórica, lo que permitirá evaluar la robustez del modelo ante escenarios reales y dinámicos del mercado.
\end{itemize}


